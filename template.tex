%%%%%%%%%%%%%%%%%%%%%%%%%%%%%%%%%%%%%%%%%
% Twenty Seconds Resume/CV
% LaTeX Template
% Version 1.0 (14/7/16)
%
% Original author:
% Carmine Spagnuolo (cspagnuolo@unisa.it) with major modifications by 
% Vel (vel@LaTeXTemplates.com) and Harsh (harsh.gadgil@gmail.com)
%
% License:
% The MIT License (see included LICENSE file)
%
%%%%%%%%%%%%%%%%%%%%%%%%%%%%%%%%%%%%%%%%%

%----------------------------------------------------------------------------------------
%	PACKAGES AND OTHER DOCUMENT CONFIGURATIONS
%----------------------------------------------------------------------------------------

\documentclass[letterpaper]{twentysecondcv} % a4paper for A4

% Command for printing skill overview bubbles
\newcommand\skills{ 
~
	\smartdiagram[bubble diagram]{
        \textbf{Full Stack}\\\textbf{Dev},
        \textbf{Frontend}\\ React Elm \\ Polymer,
        \textbf{Backend} \\ Django \\ Express,
        \textbf{Data Science} \\ Pandas \\ Scikit Learn,
        \textbf{DevOps} \\ AWS,
        \textbf{FRP} \\ RxJava RxJs
    }
}

% Programming skill bars
\programming{{C++ $\textbullet$ Swift $\textbullet$ Kotlin $\textbullet$ Elm / 1}, {Python / 3.5}, {Javascript $\textbullet$ Java / 5 }}

% Projects text
\education{
\textbf{BSc., Computer Engineering} \\
Texas A\&M University \\
2012 - 2017 | College Station, Texas
\\

\textbf {Coursera (MOOCs)} \\
\href{https://www.coursera.org/account/accomplishments/records/R2KFAFHRNEWN}{Algorithms I}, \href{https://www.coursera.org/account/accomplishments/records/LVVC3FDBZSJQ}{Algorithms II} \\
2018 | Tim Roughgarden, Stanford
\\

\textbf {Books (reading)} \\
\href{https://github.com/cyinwei/obey-the-testing-goat}{Obey the Testing Goat}  \\
\href{https://www.amazon.com/Designing-Data-Intensive-Applications-Reliable-Maintainable/dp/1449373321}{Designing Data Intensive Applications}

}

%----------------------------------------------------------------------------------------
%	 PERSONAL INFORMATION
%----------------------------------------------------------------------------------------
% If you don't need one or more of the below, just remove the content leaving the command, e.g. \cvnumberphone{}

\cvname{Yinwei (Charlie) Zhang} % Your name
\cvjobtitle{ Software Engineer} % Job
\cvcitizenship{ US Citizen }
% \aboutme{ US Citizen }
% title/career

\cvlinkedin{/in/cyinwei}
\cvgithub{cyinwei}
\cvnumberphone{(516)-949-6654} % Phone number
\cvsite{cyinwei.com} % Personal website
\cvmail{yinwei.c.zhang@gmail.com} % Email address

%----------------------------------------------------------------------------------------

\begin{document}

\makeprofile % Print the sidebar

%----------------------------------------------------------------------------------------
%	 EXPERIENCE
%----------------------------------------------------------------------------------------

\section{Experience}

\begin{twenty} % Environment for a list with descriptions
\twentyitem
	{April 2018}
	{- Present}
    {Junior Software Engineer}
    {\href{https://www.leidos.com/markets/health/hospital-systems}{Leidos, Morgantown WV}}
    {}
    {\begin{itemize}
    \item Full stack engineer for Quantum, a predictive staffing web app for employees in hospitals.
    \item Implement multiple new features, including skill mix, which filters data and feeds analytics for more fine grained staffing roles.  So combined employees now have separate analytics for nurses, and paramedics, for example.
    \item Helped migrate app backend from monolith to microservices.
    \item Lead initiative to add unit \& integration tests, documentation standards.
    \item Stack
        \begin{itemize}
            \item Frontend: Polymer SPA (ES6, HTML, CSS) in an nginx container.
            \item Backend: Microservice Jetty containers, using RxJava for async and concurrency.  Kafka for events.  Write to MongoDB for state (also Postgres).   Rest requests using RestEasy.  Data stored in HDFS / Hive for analytics (PySpark).
            \item DevOps: Deploy using AWS ECS.
            \item Workflow: Scaled Agile, two week sprints.  Git, pull requests, CI with Jenkins.
        \end{itemize}
    \end{itemize}}
    \\
	%\twentyitem{<dates>}{<title>}{<location>}{<description>}
	
\twentyitem
    {June 2015}
    {- Aug 2015}
    {Software Engineering Intern}
    {\href{https://careers.walmart.com/technology/information-technology}{Walmart, Bentonville AR}}
    {}
    {
        \begin{itemize}
            \item Developed a cloud based prototype of the next generation inventory system with streaming.  System in Java.
            \item Prototype sent item transactions from Kafka, and applied logic rules via Apache Trident, then stored results in Cassandra.
        \end{itemize}
    }
	\\
\twentyitem
    {May 2014}
    {- Aug 2014}
    {Web Developer Intern}
    {\href{https://www.sandia.gov/locations/livermore_california.html}{Sandia National Labs, Livermore CA}}
    {}
    {
        \begin{itemize}
            \item Refactored a LAMP stack web application, LSS Calls Log, that allowed lookup and selection of radioactive border detection events.
            \item Reimplemented application backend queries from MySQL to Apache Solr for faster lookups.
        \end{itemize}
    }
    \\
    
% \twentyitem
%     {May 2013}
%     {- Aug 2013}
%     {Researcher, Physics Department}
%     {\href{http://people.physics.tamu.edu/eusebi/main/index.html}{Texas A\&M, College Station TX}}
%     {}
%     {
%         \begin{itemize}
%             \item Refactored a LAMP stack web application, LSS Calls Log, that allowed lookup and selection of radioactive border detection events.
%             \item Reimplemented application backend queries from MySQL to Apache Solr for faster lookups.
%         \end{itemize}
%     }
    
    
\twentyitem
    {May 2013}
    {- Aug 2013}
    {Researcher, Physics Department}
    {\href{http://people.physics.tamu.edu/eusebi/main/index.html}{Texas A\&M, College Station TX}}
    {}
    {
        \begin{itemize}
            \item Worked with Dr. Ricardo Eusebi in high energy physics, developing an optimizer to filter signal (Higgs Boson event) from noise (other events)
            \item Implemented in C++ and with ROOT, a data analysis framework from CERN.
        \end{itemize}
    }
\end{twenty}

%----------------------------------------------------------------------------------------
%	 PROJECTS
%----------------------------------------------------------------------------------------
\section{Projects}
\begin{twenty}
\twentyitem
    {Nov 2018}
    {- Present}
    {Algo Trader}
    {}
    {}
    {
        \begin{itemize}
            \item Developing (simple) platform to read in financial data and make automated trades and test out different trading algorithms.
            \item Inputs and trades are made with Interactive Broker and Alpaca (Polygon) APIs.
            \item Stack: Python, Django.  Data read into pandas, used to evaluate rules based decision making algorithms.
            \item Collaboration with James Carnes and Norman Simon (both own the financial side).
        \end{itemize}
    }
    \\

\twentyitem
    {Oct 2018}
    {- Nov 2018}
    {Lebron vs Jordan}
    {\href{https://www.lebronvsjordan.org/}{lebronvsjordan.org}}
    {
        \begin{itemize}
            \item Landing page to ask who is the greatest basketball player ever (GOAT) Lebron or Jordan?
            \item Features polls, visualization (in PowerBI), 
            \item Build in React.  Collaboration with James Carnes (who did the visualization).
        \end{itemize}
    }

\end{twenty}

\end{document} 
